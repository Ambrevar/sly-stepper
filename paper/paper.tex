\documentclass[sigconf]{acmart}

\usepackage{blindtext} % Package to generate dummy text throughout this template 

\usepackage[english]{babel} % Language hyphenation and typographical rules
\usepackage[utf8]{inputenc}
\usepackage{booktabs} % Horizontal rules in tables
\usepackage{balance} % For balanced columns on the last page
\usepackage{lettrine} % The lettrine is the first enlarged letter at the beginning of the text

\author{João Távora} \affiliation{Ravenpack}
\email{joaotavora@gmail.com}
\acmConference{2020 European Lisp Symposium}{April 27th - April 28th 2020}{Zurich, Switzerland}


\setcopyright{rightsretained} % if you give the rights to ACM
% \acmDOI{...} % DOI - Insert your DOI below...
% \acmISBN{...} % ISBN - Insert your conference/workshop's ISBN below...
\acmYear{2020} % Insert Publication year
\copyrightyear{2020} % Insert Copyright year (typically the same as above)
\acmConference[ELS 2020]{2020 European Lisp Symposium}{April 27th - April 28th 2020}{Zurich, Switzerland}

\begin{document}
\title{A portable, annotation-based, visual stepper for Common Lisp}

\begin{abstract}
  Many programming systems feature a stepping debugger, a tool that
  lets users execute code, section by section, in steps of their own
  choosing.  Despite many attempts throughout the decades, the Common
  Lisp language is still lacking in this regard.  We propose and
  describe the workings of a new, portable, visual stepping facility
  for Common Lisp, realized as an extension to SLY, a
  cross-implementation Common Lisp IDE for the Emacs editor.  This
  facility is realized as an increment to an existing source code
  annotation system known as ``stickers'', whose working principles we
  also describe in this work.  As part of the solution arrived at for
  the main objective, we also present two reusable software
  components: (1) a simple, near portable technique for constructing a
  source-tracking Common Lisp expression reader in terms of a
  preexisting compliant expression reader and (2) a technique to carry
  over source-tracking information to the expansion of macro
  expressions.
\end{abstract}

\keywords{Debugging, Stepping, Common Lisp}

\begin{CCSXML}
<ccs2012>
   <concept>
       <concept_id>10011007.10011006</concept_id>
       <concept_desc>Software and its engineering~Software notations and tools</concept_desc>
       <concept_significance>500</concept_significance>
       </concept>
 </ccs2012>
\end{CCSXML}

\ccsdesc[500]{Software and its engineering~Software notations and tools}

\maketitle
\section{Introduction}
\subsection{What is stepping}

A program stepper lets users execute code, section by section, in
steps of their own choosing.  A number of hidden control points are
inserted along the program's execution paths.  At each point, the
system may interrupt the program and wait for user instructions before
and/or after executing the next section.  Steppers often fall within
the category of \emph{program execution monitors}, along with
profilers, tracers and other debugging tools.  Stepping is the one of
the most popular forms of debugging since it allows users to study the
evolution of a program's state by direct inspection, as opposed to a
combination of conjecture and experimentation.  A stepper gives users
full control over the speed of the actual program and often even
allows manipulations of the program's state.  This frequently presents
a debugging advantage over guessing the program's state or monitoring
its outputs.

Many programming systems, indeed all of the most popular programming
systems provide stepper tools as a part of a debugging tool-chain,
i.e. a set of programs often developed and distributed alongside
language compilers and interpreters.  In languages such as C, the
prevalence of a stepping system (such as the popular \texttt{gdb}
program) is so strong that it becomes synonymous with the term
``debugger''. This sometimes leads to users migrating to the Lisp
family of languages being surprised to find debugging systems that
aren't about stepping at all.

Stepper systems usually function through text-based interfaces that
are capable of displaying the source code of the relevant code
sections.  Nevertheless, many users of steppers tool prefer to use
them through the more sophisticated user interfaces of editor programs
and integrated development environments (IDEs).  We shall call these
\emph{visual} steppers\footnote{After the nomenclature used in
  \cite{annotation-based}}: special-purpose programs running inside
the IDE communicate with the stepper tool by means of a protocol and
visually annotate the program's source text with the results of the
stepping session.  The expression about to be executed may e.g. be
marked with a red dot or highlighted in a special color.  Furthermore,
a modern user's expectations of a stepper system might include the
ability to add break points; to ``step over'' an expression; to ``step
into'' a call to a function defined elsewhere in the srouce code; to
``step out of'' the current call; to ``continue'' to a certain point,
and to inspect the values of local and global variables by
mouse-clicking on their manifestations in the source code.

\subsection{Common Lisp stepping}

% https://groups.google.com/forum/#!topic/comp.lang.lisp/nj3jFxcJYM0

Common Lisp programmers are so detached from the practice of stepping
that some will simply declare that's proof that they don't need one at
all.  There is a hint of truth to this, as Common Lisp systems have
traditionally directed efforts to other types of debugging facilities
such as powerful interactive program restarts and function traces.  At
any rate, it seems undeniable that, stepper or no stepper, Common Lisp
programs will be debugged.

Nevertheless, an abundance of Common Lisp steppers have been proposed
throughout the decades, even if few have actually enjoyed any
adherence.  Anecdotal evidence suggests potential newcomers shy away
from Common Lisp because of missing stepper functionality.  Indeed,
this reality appears not to be lost even on the authors of the Common
Lisp specification, which have included in their work a provision for
a special implementation-defined \texttt{CL:STEP} macro.

Some Lisp implementations can and do implement stepping to various
levels of ability.  SBCL combines its implementation of
\texttt{CL:STEP} with the restarts system to provide a text-based
stepper, while the LispWorks\cite{lispworks-stepper} and Allegro CL
systems have sophisticated graphical stepping dialogues with views to
the source code and program state.  Unfortunately, these systems are
unavailable to whomever wishes to try portable programs across
different Common Lisp implementations.  Even if one switches between
two implementations that do have steppers, the difference in
capability and user interface is often enough to discourage either
system's use.

The problem of interface inconsistency across implementations is not
exclusive to the stepper feature: other debugging tools such as the
inspector, the debugger or the REPL suffer from it.  Thus, many Common
Lisp programmers will use a generic text editor such as GNU
Emacs\cite{emacs} together with the SLIME\cite{slime} extension.  This
combination forms a capable, implementation-agnostic Common Lisp IDE
that suppresses the problems described above and provides a consistent
user interface to many debugging features.

Regrettably, even though Emacs provides visual stepper interfaces for
many programming languages, SLIME doesn't provide a portable stepper
interface.  We believe this to be due to the fact that the technical
challenges to be surmounted are greater than for other debugging
tools.  Among other problems, the portability mandate of SLIME implies
it is ultimately only allowed to invoke functionality present in all
the Common Lisp systems it connects to.  This means that even if all
implementations where to implement some form of \texttt{CL:STEP}, that
alone wouldn't be powerful enough to, say, communicate source location
information to and from Emacs, annotating the program source.

\subsection{A portable, visual stepper for Common
  Lisp}

We should note that none of the obstacles listed above are conceptual
in nature, so there must still be hope for a portable, visual stepper
for Common Lisp. The SLIME/Emacs combination makes it a particularly
attractive target for such a tool, given its relevance among Common
Lisp users and the flexibility of Emacs's Elisp language.

In fact, the SLY Common Lisp IDE\cite{sly}, a derivative program of
SLIME, has redesigned some of its predecessor's underpinnings to make
this even easier.  We shall describe how ``stickers'', a feature that
SLY has recently acquired, lets users manually instrument selected
Common Lisp forms whose results they are interested in.  Stickers are
already a ``poor man's stepper'', in the sense that they have some
fundamental semantics of stepping but still encumber the user with
work than could be performed automatically.  To fill this gap, we
shall explore methods of combining stickers with automatic code
analysis.  We shall then be able to present an innovative stepper tool
for the Common Lisp language based on the SLY extension to the Emacs
editor, hereafter designated the \emph{SLY/Emacs stepper}.

% Cue SLY pic

The rest of this essay is structured as follows: we start by reviewing
related work, to establish better context for the difficulties
encountered and decisions made.  We'll then describe the three main
components of the proposed SLY/Emacs stepper: (1) the interface
between SLY/Emacs stickers and Common Lisp; (2) a source-tracking form
reader; and (3) a manner of macroexpanding code that maintains the
memory of source-tracked forms.  We shall finish with a discussion of
the stepper tool's interface, its limitations and future work.

\section{Related work}

``Stepping is an old idea.''  So go the opening words of this
section's namesake in the article ``Annotation-Based Program
Stepping'', written by MIT's G. Parker in 1987\cite{annotation-based}.
In this article, the author surveys the 1970 decade's efforts to
develop various kinds of stepping tools in the MACLISP environment.
Likewise, we shall proceed to evaluate a small sample of Common Lisp
stepper systems, focusing on the ones that are portable, visual and
intersect our methodology.

In the remainder of Parker's paper\cite{annotation-based}, a
visual stepper system, VisiStep, is described.  Its distinguishing
characteristics are the integration with the MACLISP system and an
\emph{annotation-based} approach, a key difference to other
\emph{evaluator-based} techniques.

Annotation-based program stepping is a form of code instrumentation.
It comprises the addition of statements to the program shortly before
its compilation.  By way of a so-called \emph{wrapper} macro, these
statements are added before and after each section to be stepped.  The
addition is transient and invisible: it does not modify the source
file.  Furthermore, the program cannot itself discern the presence or
absence of these additional statements, so its outputs are unmodified.
Parker\cite{annotation-based} points out that this approach can work
with an unmodified evaluator, since the compiler is simply given more
instructions to compile.  He also asserts this approach to be more
efficient, more portable and more selective, the latter meaning that
it allows the user to select only those sections of the code that he
wishes to step through.  However, the author acknowledges the
annotation-based stepper's difficulty in handling some macro's
non-evaluated syntactic elements (such as the arguments to
\texttt{COND}), and how it must rely on ``code-walking
knowledge''\cite[I-4.8]{annotation-based} to determine the forms where
the wrapping may take place.

By contrast, an evaluator-based or \emph{interpretative} approach
involves writing a Lisp evaluator or instrumenting the Lisp
interpreter. The evaluator itself then becomes responsible for issuing
the stepper-enabling statements before and after each evaluation.
``UniCStep - a Visual Stepper for COMMON LISP'', written by I. Haulsen
and A. Sodan in 1989\cite{evaluation-based}, presents such a stepper
system, written for an early version of the GNU Emacs editor. The
authors reply directly to Parker's contention's of the superiority of
the annotation-based approach, asserting the evaluator-based approach
to be more comfortable and flexible because the user does not have to
specify in advance what to step and where to stop.  They assent to one
technical disadvantage such as the fact that evaluator-based
alternative need a loader to be emulated and more sophisticated ways
of remembering the source of the loaded code.

We should note a more recent attempt at a Common Lisp stepper, such as
Pascal Bourguignon's work\cite{bourguignon}.  This consists of a
portable, evaluator-based approach that replicates the
implementation-defined behavior of Common Lisp's \texttt{STEP} macro.
Bourguignon's stepper provides a replacement package for the standard
\texttt{COMMON-LISP} package, through which the user must re-load the
code whose forms can then be passed to the \texttt{STEP} macro.  This
stepper has no editor or source-tracking integration as of yet, but it
seems to have been in the plans at some point during its development.

Finally, a word should be spared for Emacs's \texttt{edebug.el}
authored ca. 1988 by Daniel LaLiberte\cite{edebug}.  Edebug is
designed to step through Emacs Lisp programs within Emacs itself.
Since it executes inside the Lisp machine that is also the editor, the
source-tracking integration is very good.  \texttt{edebug.el} is an
annotation-based stepper that deals with the problem of a macro's
un-evaluated syntactic elements by skipping macros it knows nothing
about.  The macros whose expansions the user is interested in can be
annotated separately with \texttt{edebug.el}-specific declarations.

\section{Methods}

Our proposed portable stepper system for SLY/Emacs can be
broken down into three main components:

\begin{enumerate}
\item A non-intrusive source code annotation system, called
  ``stickers''. This system primarily allows ``interesting'' Common
  Lisp forms to be designated by the user.  On compilation, the
  annotated code is transmitted to the Common Lisp compiler, and
  executes equivalently to non-annotated code;
\item A \emph{source-tracking reader}, i.e. a process by which a
  stream of characters containing source code forms is read into a
  symbolic expression representing the form, whilst recording the
  positions of the start and end characters of each sub-form;
\item A \emph{specialized code walker}, a process by which an
  arbitrary Common Lisp form can be traversed at compilation-time to
  determine the syntactic value of each of its sub-forms as processed
  by the compiler after the macro-expanding phase.
\end{enumerate}

It should become apparent that the application of 3. to the results of
2. relieve users in 1. of the need to manually designate forms of
interest.  They sole job becomes requesting the annotated compilation
of arbitrary lengths of source code, leaving the stepper system to
automatically annotate all possible forms of interest.

The following subsections detail the workings of each component in
this arrangement.

\subsection{Stickers}\label{stickers}

Stickers are a form of code annotation in use in SLY/Emacs.  Initially
conceived as an alternative to the \texttt{PRINT} or \texttt{FORMAT}
statements introduced by users when debugging programs, this system
lets users visually mark individual symbolic or compound forms in
whose future execution they're interested in.  Crucially, the visual
markings exist only in Emacs's memory for as long as the user
wishes. They aren't saved in the source code itself.  When the
compilation of the containing top-level form happens from within
SLY/Emacs, SLY will collect those visual markings, enumerate them, and
emit for compilation a modified version of the form.  This process is
called \emph{arming the stickers}.

The modified version is functionally equivalent to the original in the
sense that i.e. user programs have no way to detect which one they are
executing.  The modifications consist of multiple invocations of a
special \texttt{RECORD} wrapper macro\footnote{This macro is very
  similar to the \texttt{WRAP} macro presented in
  \cite{annotation-based}, which the exception that for some technical
  reason that other version was realized as a special form.}, whose
definition is presented below in much simplified fashion:

\begin{verbatim}
(defmacro record (id &body body)
  `(let ((%retval :exited-non-locally)
         (%condition)
         (%sticker (find-sticker ,id)))
     (handler-bind ((condition (lambda (c)
                                 (setq %condition c))))
       (before-sticker %sticker)
       (unwind-protect
            (values-list
             (setq %retval (multiple-value-list
                            (progn ,@body))))
         (after-sticker %sticker %retval %condition)))))
\end{verbatim}

E.g. if the user marks the forms \texttt{(foo (bar))} and
\texttt{(bar)} inside the following expression:

\begin{verbatim}
(let ((baz 42)) (+ (foo (bar)) baz))
\end{verbatim}

The sticker system will collect the two markings, label them with the
numbers 1 and 2 and emit the following equivalent form for
compilation:

\begin{verbatim}
(let ((baz 42)) (+ (record 1 (foo (record 2 (bar)))) baz))
\end{verbatim}

As can be seen, every time the expression above is executed, expansion
of the wrapper macro causes the functions \texttt{before-sticker} and
\texttt{after-sticker} to be called with the appropriate
\texttt{\%sticker} object.  Depending on the user's preference, these
functions may decide to stop execution of the program (by way of
\texttt{invoke-debugger}), or to simply record the fact that the
sticker was traversed.  The list of recordings can later be retrieved
and replayed later inside SLY/Emacs.

In our simple example, the benevolent user placed stickers on two
expressions that are indeed executed, i.e. they exist in places of
evaluation as defined by the syntax of the \texttt{(let ...)}  special
form and the \texttt{(+ ...)} function form.  If a sticker had instead
been placed on the expressions \texttt{((baz 42))} or \texttt{+} --
two examples of non-evaluated forms -- that would have created a
difficulty, since the code would become syntactically invalid and fail
compilation.  In a worse situation, the code would still be
syntactically valid but semantically absurd.

SLY/Emacs's sticker system doesn't have a way to reject these
individual stickers, so it proceeds heuristically: it rejects the
compilation of the whole top-level form if it determines that arming
stickers results in an \emph{increase} to the number of compilation
warnings.  In that case, the original form is compiled but the
stickers \emph{fail to arm}.  This strategy works for a vast majority
of cases, but it doesn't seem impossible to construct a pathological
case where the principle of functional equivalence stated above is
violated.

\subsection{Source-tracking form reader}\label{st-reader}

At its simplest definition, a source-tracking form reader is a
variation of the Common Lisp \texttt{CL:READ} function that invokes a
hook every time a sub-form is read, and proceeds to pass to this hook
a measure of the character distance traveled so far to read it.  This
enables programs that need both the usual results of \texttt{CL:READ}
and a table of form-to-source-code pairings.

Since the Common Lisp standard doesn't specify any form of
source-tracking reader, some preexisting alternatives were evaluated:

\begin{enumerate}
\item \texttt{Eclector}\cite{eclector}, a self-described ``portable
  Common Lisp reader that is highly customizable and can return
  concrete syntax trees'', is a full realization of a Common Lisp code
  reader that doesn't rely on the implementation's.  Its
  distinguishing characteristic are ``concrete syntax trees'' that
  aren't represented by \texttt{CONS} cells, rather by CLOS objects
  that mimic the properties of such cells while also keeping
  ``concrete'' source file information;

  %% https://hub.darcs.net/hu.dwim/hu.dwim.reader/browse/source/reader.lisp
\item \texttt{hu.DWIM.reader}\cite{bourguignon-reader}, by Pascal
  Bourguignon, is another full realization of a compliant, portable
  and programmable Common Lisp reader.  Though not a source-tracking
  reader \emph{per se}, it could be used in conjunction with a
  mechanism to track character counts in streams;
\end{enumerate}

Any alternative could have been used for our purposes (the second one
with minor changes).  However, since we wish to minimize our program's
dependency chain we also searched for simpler alternatives.

We reasoned that our stepper tool already expects a compliant Common
Lisp implementation.  Therefore it may, by definition, also expect a
compliant \texttt{CL:READ}.  So we set out to design a portable
source-tracking reader that completely reuses the implementation's
reader instead of replacing its implementation entirely.  To achieve
this, we settled on an arrangement of two separate techniques:

\begin{itemize}
\item A \emph{character counting stream}.  This wraps the input
  \texttt{CL:STREAM} object (from which we intend to \texttt{CL:READ}
  from) inside a so-called ``gray stream'' object.  Such objects are
  not in the standard but are still widely available and used
  extensions to the Common Lisp standard\footnote{They are already
    heavily used in SLY/Emacs, for example}.  Gray streams allow the
  individual character reading operations to be intercepted and
  controlled by the user.  In this case, our character counting stream
  is equivalent to the wrapped input stream except for the fact that
  it keeps count of the number of characters read so far.
  
\item A \emph{substitute read-table}, achieved by rebinding the
  variable \texttt{*READTABLE*}.  This table \emph{shadows} each of
  the entries of the current read-table (using
  \texttt{GET/SET-MACRO-CHARACTER}) with a function that allows full
  control of each characters influence in the returned symbolic
  expression.  By setting up this function in a particular manner, the
  resulting table can be made functionally indistinguishable from the
  original one, while retaining the ability to invoke a user's hook to
  record source positions.
\end{itemize}

The inter-operation of the two techniques is summarized by the
\texttt{READ-TRACKING-SOURCE} function:

\begin{verbatim}
(defun read-tracking-source
    (&optional (stream *standard-input*)
       (eof-error-p t) eof-value
       recursive-p (observer #'ignore))
  (let* ((ccs (char-counting-stream stream))
         (*readtable*
           (substitution-table
            *readtable*
            (lambda (shadowed-entry)
              (let (;; correct for the fact that
                    ;; one character of the form
                    ;; has already been read.
                    (start (1- (char-count ccs)))
                    (results (multiple-value-list
                               (funcall shadowed-entry)))
                    (end (char-count ccs)))
                (multiple-value-prog1 (apply #'values results)
                  (when results
                    (funcall observer (car results)
                             start end))))))))
    (read ccs eof-error-p eof-value recursive-p)))
\end{verbatim}

This mechanism provides a simple, reliable\footnote{This was tested in
  SBCL, Allegro CL, CCL and ECL.  There are differences to the way
  that some implementations will construct the standard read-table
  (for example, SBCL and Allegro CL represent ``constituent''
  characters differently) and \texttt{SUBSTITUTION-TABLE} has special
  provisions for that.  The same technique should in theory work with
  non-standard read-tables, but this has not been tested.} and
portable\footnote{As noted, except for the use of gray streams.}  way
to read source-code.  We can demonstrate its use on the simple
\texttt{(let ...)} form already presented above:

\begin{verbatim}
(with-input-from-string
    (s "(let ((baz 42)) (+ (foo (bar)) baz))")
  (read-tracking-source
   s t nil nil
   (lambda (form start end)
     (format t "~&(~2,a ~2,a) <=> ~a"
             start end form))))
\end{verbatim}

This expression returns the intended symbolic expression representing
the form, \texttt{(LET ((BAZ 42)) (+ (FOO (BAR)) BAZ))}, while also
producing the desired table of source positions:

\begin{verbatim}
(1  4 ) <=> LET
(7  10) <=> BAZ
(11 13) <=> 42
(6  14) <=> (BAZ 42)
(5  15) <=> ((BAZ 42))
(17 18) <=> +
(20 23) <=> FOO
(25 28) <=> BAR
(24 29) <=> (BAR)
(19 30) <=> (FOO (BAR))
(31 34) <=> BAZ
(16 35) <=> (+ (FOO (BAR)) BAZ)
(0  36) <=> (LET ((BAZ 42))
              (+ (FOO (BAR)) BAZ))
\end{verbatim}

\subsection{Specialized code walker}

Equipped with a correspondence between forms and source code, the
Common Lisp side of our stepper system nears a state where it may
inform SLY/Emacs of where to place sticker annotations.  A final
obstacle remains: as we have seen in \ref{stickers}, only a subset of
these forms may be annotated with the \texttt{RECORD} macro, i.e. we
are only interested in the ones that are executable.

Clearly, we need an agent that understands the semantics of each
Common Lisp special form\footnote{The human programmer \emph{is} such
  an agent, but he is precisely the one we are trying to relieve of
  these tasks.}, determines sub-expressions of interest in our
source-mapped tree and discards all the others.  As was already noted
in \cite{annotation-based}, this seemingly simple task is severely
complicated by macros and by the specific constraints of a stepper
system.

\subsubsection{Mnesic macroexpansion}

To reach a state where nothing but special and function forms exist, a
macroexpander must remove macros by expanding them.  However, in doing
so, our program \emph{must also remember} whence each macro's
expansion came, specifically the source position of the form in its
pre-expansion state.  This what we mean by \emph{mnesic
  macroexpansion}, the opposite of \emph{amnesic macroexpansion}

Take the form:

\begin{verbatim}
(LET ((BAZ 42))
   (COND ((PLUSP BAZ) (FOO)) (T (BAR))))
\end{verbatim}

It may be expanded to something like\footnote{This expansion is
  SBCL's.}:

\begin{verbatim}
(LET ((BAZ 42))
  (IF (PLUSP BAZ)
      (FOO)
      (THE T (BAR))))
\end{verbatim}

By this point, the system may finally come to the realization that
only the forms \texttt{42}, \texttt{baz}, \texttt{(plusp baz)},
\texttt{(foo)}, \texttt{(bar)}, \texttt{(if ...)} and \texttt{(let
  ...)} are in positions of evaluation.  Regrettably, it may now have
lost track of \emph{where} each form lives in the source.

To recover this information, it is not enough to naively consult the
hash-table produced in \ref{st-reader}, since some forms didn't exist
in our original source-tracked version.  Even if they did, the
macroexpanding facilities are not generally obliged to return the same
\texttt{CONS} objects for the forms, regardless of whether they expand
them or not.

In the \texttt{Eclector} library discussed in section \ref{st-reader},
a \texttt{RECONSTRUCT} function attempts to solve this very problem by
correlating the fully macro-expanded ``raw'' tree with the original
``concrete syntax tree'', returning a mirroring of the former that
keeps as much from the latter as possible.  After some experimentation
with this approach, we noticed it missed many forms in executable
positions and so decided it wasn't producing the results we had hoped
for.  Moreover, the quality of results tended to vary across
implementations, possibly due to the aforementioned
\texttt{CONS}-related problems.

To overcome this obstacle we need a different approach.  Instead of
trying to recover from a fully macroexpanded tree, we must hook into
macroexpansion as soon as it happens.  This shall allow us to lose as
little source-tracking information as possible.  Thus, we conclude
that a programmable, portable code-walker is necessary.  Such a system
shall let us execute a hook at each macroexpansion step shortly before
and shortly after each expansion.

After surveying open-source alternatives for code-walkers, we settled
on a program called \texttt{AGNOSTIC-LIZARD}\cite{agnostic-lizard},
which fits exactly these requirements\footnote{Furthermore,
  \texttt{AGNOSTIC-LIZARD} contains useful provisions to shield user
  code from certain nonconformities in the macroexpansions of certain
  built-in macros, such as \texttt{DEFUN}.}.  Its main primitive,
\texttt{AGNOSTIC-LIZARD:WALK-FORM}, produces the desired full
macroexpansion, and can also be given a set of callback functions as
hooks.

Here's the snippet that illustrates our use of this walker:

\begin{verbatim}
(defun mnesic-macroexpand-all (form subform-positions)
  (let (stack (expansion-positions (make-hash-table)))
    (values
     (agnostic-lizard:walk-form
      form nil
      :on-every-form-pre
      (lambda (subform env)
        (push (list
               :from subform
               :at (gethash subform subform-positions))
         stack)
        subform)
      :on-every-form
      (lambda (expansion env)
        (push (pop stack)
              (gethash expansion expansion-positions))
        expansion))
     expansion-positions)))
\end{verbatim}

As we can see, our \texttt{MNESIC-MACROEXPAND-ALL} function uses a
stack to take advantage of the manner in which macroexpansion
traverses the tree: there may be more than one consecutive call to
each of the callbacks \texttt{:ON-EVERY-FORM-PRE} and
\texttt{:ON-EVERY-FORM}.  However, in the end, the calls to one and
the other perfectly mirror each other.  Each element of the stack
holds the result looked up in the \texttt{SUBFORM-POSITIONS}
hash-table for non-expanded forms.  That source information is later
saved on the output hash-table \texttt{EXPANSION-POSITIONS}, whose
keys are of expanded forms.

If we give this function the form:

\begin{verbatim}
(COND ((FOO) (BAR)) ((BAZ) (QUUX)) (T 42))
\end{verbatim}

We may obtain something\footnote{This is Allegro CL's
  expansion.  SBCL's is much simpler, and thus not so good for
  illustrative purposes.} like:

\begin{verbatim}
(IF (FOO) (PROGN (BAR))
    (IF (BAZ) (PROGN (QUUX))
        (IF T (PROGN 42) NIL)))
\end{verbatim}

The resulting hash-table \texttt{EXPANSION-POSITIONS}, returned as a
second value, has these mappings:

\begin{verbatim}
(IF (FOO) ..) => (:from (COND ((FOO)..)) :at (0 . 42))
42            => (:from 42               :at (38 . 40))
(BAR)         => (:from (BAR)            :at (13 . 18))
(PROGN 42)    => (:from (PROGN 42)       :at NIL)
(PROGN (BAR)) => (:from (PROGN (BAR))    :at NIL)
T             => (:from T                :at (36 . 37))
(IF T ...)    => (:from (COND (T 42))    :at NIL)
(QUUX)        => (:from (QUUX)           :at (27 . 33))
NIL           => (:from NIL              :at NIL)
(IF (BAZ) ..) => (:from (COND ((BAZ) ..)):at NIL)
(BAZ)         => (:from (BAZ)            :at (21 . 26)))
(FOO)         => (:from (FOO)            :at (7 . 12)))
(PROGN (QUUX))=> (:from (PROGN (QUUX))   :at NIL)
\end{verbatim}

As can be seen, numerous new forms appeared in the expansion, but
\texttt{MNESIC-MACROEXPAND-ALL} succeeded in keeping the source
information information for all of the relevant ones.

\subsubsection{Annotating interesting forms and putting it all
  together}\label{collecting-interesting-forms}

A final piece of the puzzle is needed.  A function named
\texttt{FORMS-OF-INTEREST} is to be given the fully macroexpanded tree
and along with the source-tracking information for that tree.  Its
task is to traverse the tree while looking for each of the 25 Common
Lisp special compound forms\footnote{In reality, a few macros like
  \texttt{COND} and \texttt{DEFUN} are left unexpanded by
  \texttt{AGNOSTIC-LIZARD} so they are analysed separately as well},
considering the evaluation rules of each.  Unknown forms are assumed
to be function calls, whose evaluation rules are equally well known.
For each sub-expression in a position of execution, the source
location is looked up and the form is collected, so that it can later
be reported to SLY/Emacs's sticker system for annotation.  Though its
listing is too large to include here, its implementation is
straightforward but for one detail described in section \ref{atoms}.

As we are nearing the end of our journey, we can now start putting all
the pieces together. The following snippet is the final form of our
Common Lisp function. Its results can be handed to SLY/Emacs for
instrumentation through stickers as described in section
\ref{stickers}.  After compilation, the instrumented code is now
steppable.

\begin{verbatim}
(defun stepper-sticker-locations (string)
  (with-input-from-string (stream string)
    (let* ((form-positions (make-hash-table))
           (form-tree
             (read-tracking-source
              stream nil nil nil
              (lambda (form start end)
                (setf (gethash form form-positions)
                      (cons start end))))))
      (multiple-value-bind (expanded-tree
                            expansion-positions)
          (mnesic-macroexpand-all form-tree
                                  form-positions)
        (forms-of-interest
         expanded-tree expansion-positions)))))
\end{verbatim}

\section{Results and further work}\label{results}

We have released the result of our work on the GitHub
platform\footnote{See
  \texttt{https://github.com/joaotavora/sly-slepper.git}.}.

From an end user's perspective, to put the new SLY/Emacs stepper to
work means pressing the key chord \texttt{C-c C-s P} (\emph{control-c,
  control-s, capital P}) while the cursor is on a top-level form.
This causes the interesting sub-forms of that top-level form to be
automatically decorated with sticker \emph{overlays}, which by default
uses different shades of the color gray.  As described in
\ref{stickers}, a posterior compilation of that same top-level form
shall \emph{arm} the stickers and convert the overlays' color to
shades of blue.  From this point on, the stickers are executed as soon
as the user arranges for the instrumented code to be run as usual.

Note that the default behavior of the sticker system doesn't equate
the execution of an instrumented form to a break point, i.e. the
invocation of the Lisp debugger.  This is by design.  As was explained
in section \ref{stickers}, the default behavior is to have sticker
executions merely record the return values (or non-local exits) for
later replay.  This which can be achieved with the key chord
\texttt{C-c C-s C-r} or via \texttt{M-x sly-stickers-replay}.
Alternatively, the key chord \texttt{C-c C-s S} (or \texttt{M-x
  sly-stickers-fetch}) can be used to fetch the most recent recordings
for each sticker and visually decorate the source code, indicating (1)
stickers that have been executed; (2) those that haven't yet, and (3)
those that have exited non-locally.

Finally, to enable the classic stepper functionality, the user must
explicitly select ``breaking stickers'' by affecting the value of the
\texttt{SLYNK-STICKERS:*BREAK-ON-STICKERS*} variable.

We shall see, as we discuss its limitations, that the resulting
SLY/Emacs stepper tool is still in its infancy.  It can nevertheless
be said to work reasonably well for a majority of normal
circumstances, succeeding in instrumenting forms effectively and
efficiently, while providing satisfactory methods of navigation among
stickers.

\subsection{Limitations concerning atoms}\label{atoms}

In section \ref{collecting-interesting-forms}, we sidestepped a
notorious difficulty with atomic forms, i.e. one-symbol symbolic
expressions.  This class of difficulties is already alluded to in
\cite[I-4.8]{annotation-based}.  The problem with atoms can be
observed with the simplest of forms:

\begin{verbatim}
(lambda (x) x)
\end{verbatim}

In this example, we note that the atom \texttt{X} has two different
\emph{manifestations} in the encompassing form.  Naturally one wants
only the latter to be annotated, and not the first.  However, that is
hard to determine reliably since both are represented by the very same
object.  This is in stark contrast to compound forms represented by
different \texttt{CONS} cells.

We can enhance the form/position pairings table used above to record
the fact that there is more than one manifestation of an atom, but
that's not enough to know in \texttt{MNESIC-MACROEXPAND-ALL} which of
those is a in a position of execution.  The reason is that the
\texttt{AGNOSTIC-LIZARD} macroexpander will only traverse sub-forms of
forms actually returned by a macro's expansion.  In this example, our
hook is only called on the \texttt{(lambda (x) x)} and \texttt{x}
forms, \emph{not} on the \texttt{(x)} form.  The latter form is merely
an argument to the macro itself where we have no power of
intervention, and thus there is no easy way to invalidate the first
manifestation.

However, since we \emph{do} know that \texttt{x} is manifested at
\texttt{(9 . 10)} and \texttt{(12 . 13)} it is possible to devise
heuristics to trace back to the knowledge gathered when first reading
the form and traverse the atom's parent forms, given only the atom.  A
very simple heuristic can proceed like this: if the atom exists inside
a compound form that does not occur in the final expansion, then that
atom isn't interesting, otherwise, it is.  This appears to solve the
above situation but fails miserably in the presence of the
\texttt{LOOP} macro since this macro has all the variable definition
``unprotected'' by parenthesis.

Hard-wiring exceptions to \texttt{LOOP} and other macros could
ameliorate the situation, but overall this strategy feels murky and
insufficient.  On the other hand, if more aggressive strategies of
atomic annotation are attempted, the SLY/Emacs sticker system has
already shown to be reliable in the sense that if it needs to fail
(because of an incorrect form being annotated), it will mostly do so
early.  Thus the potential to mislead users to wrong debugging
conclusions is minimized.

A better approach to solve this problem could revisit
\texttt{Eclector}'s ``concrete syntax trees'' or a variation thereof
and use a portable, programmable macroexpander that also understands
these types of trees, where different manifestations of the same
atomic form are represented by different objects.

For now, the proposed SLY/Emacs's stepper works around this limitation
by behaving conservatively and only annotating atoms in positions that
are guaranteed to be safe, such as inside function call forms.  This
makes for the majority of situations in practice.  Furthermore, users
can always manually add stickers to other atom manifestations they are
interested in and know to also be safe.

\subsection{Interface limitations}

As seen in section \ref{stickers}, SLY/Emacs's sticker system has no
notion of a stack: all the armed stickers are enumerated serially and
thus hierarchically equivalent.  Therefore, the common ``step in/step
out/finish'' functionality of common steppers is unavailable as such.
Once stepping has been initiated, it is currently not possible to
``step over'' arbitrarily large sections of uninteresting code, nor is
it yet possible to designate a sticker to continue to.  The nearest
thing available is the possibility to ignore a particular sticker
number.  Furthermore, there is as of yet no notion of a stepping
``session'': once armed, stickers take effect immediately and stay
armed (even if the corresponding source code is deleted) until the
definition they pertain to is compiled again without stickers.

These features don't seem hard to realize.  E.g. to enable more
sophisticated navigation behavior behavior the \texttt{RECORD} macro
could use a special variable to be made aware of recursive invocations
to itself or to the currently executing stack frame.  Thus we could
keep track of stickers being executed \emph{inside} each other,
i.e. within the dynamic scope of a previously active sticker
annotation.

\section{Discussion}

Common Lisp users are concerned about features that facilitate
day-to-day development.  \texttt{TRACE}, REPLs, \texttt{PRINT}
statements, the interactive debuggers, profilers and stickers are all
ways to solve certain debugging problems: some are more suitable to
some situations than others.  Therefore, it's important to note that
program stepping is just another tool in the toolbox, not a
panacea.

By the same token, if one does implement such a tool, it should be
done in a manner that is of actual, practical use.  This was the
thinking behind the SLY/Emacs stepper.  We think it especially
fortunate that annotation-based approach and the manner of source-code
correlation in SLY/Emacs's stickers feature doesn't make use of direct
references to source files or file positions.  As was shown in
\ref{stickers}, we merely keep an enumerated list of stickers
identifiers synchronized between Common Lisp and SLY/Emacs, a
mechanism that is simple but effective.  It adequately resists some
modifications to the source of the instrumented form or its
whereabouts, such as moving it around in the source file, or even
adding white-space and comments inside it.  This detail is crucial in
making stickers and stepping actually usable for day-to-day
programming, since the user isn't dragged away from his editor or
forced to a special confinement while stepping.  We note that these
advantages of annotation-based steppers were already hinted at back in
1989 by the proponents of evaluation-based steppers\cite[l.41,
l.42]{evaluation-based}.

By contrast, an hypothetical evaluation-based visual stepper (such an
as enhanced version of \cite{bourguignon}) would find it difficult to
maintain this advantage since the source-code correlation is achieved
at evaluation-time and is harder to mutate effectively afterwards.
Perhaps this fact can explain why the non-portable visual steppers of
the Allegro CL and LispWorks implementations don't work directly with
the source-editing facilities present in these
IDEs\footnote{LispWorks's stepper manual specifically mentions this
  fact in \cite{lispworks-stepper}.}.  A possible solution would be to
represent source-code correlation in terms of sub-expression paths in
the form tree instead of character positions, but it would still be
hard to resist deletions and insertions at top level.  A more
heavy-handed solution would lock the source file read-only for the
duration of the stepping session.  However, users are normally adverse
to such confinements.

We also think it fortunate that stepping is implemented as an
increment to the existing stickers functionality.  As in
\cite{annotation-based}, we consider it an advantage of the
annotation-based stepping systems that users are allowed to instrument
only the definitions they are interested in.  Indeed, it is often the
case that stepper systems become tedious to operate because they step
on \emph{too much}.  Our system has an additional selectivity
advantage: since the annotations are stickers, users can manually
adjust the automatically placed stickers, removing the ones they are
not interested in, or adding others.

It may also be noted that the usual stepping paradigm where the
program is stopped at each point is only one of the possibilities
afforded by stickers.  As we explained in sections \ref{stickers} and
\ref{results}, two further ones can constitute innovative means of
debugging: the \emph{post-mortem} replay of sticker recordings and the
visual decoration of source code with colors to indicate the most
recent state of each sticker's execution.

Its current limitations notwithstanding, the SLY/Emacs stepper
described in this essay is the most effective example of a portable,
visual stepping facility for the Common Lisp ecosystem.  In
formulating its development we have also described (1) a simple,
portable technique for constructing a source-tracking reader in terms
of a compliant reader implementation and (2) a reliable technique to
carry over source-tracking information to macroexpansion.  It is
conceivable that other debugging tools could be constructed from
either of these elements.

\section{Acknowledgements}

We would like to thank Luís Oliveira and Michael Kirkland for early
comments and advice when discussing the stepper feature, as well as
Andrew Lawson for the encouragement to produce this work.

%----------------------------------------------------------------------------------------
%	REFERENCE LIST
%----------------------------------------------------------------------------------------

\bibliographystyle{ACM-Reference-Format}
\bibliography{paper}

\end{document}



% LocalWords:  macroexpander macroexpansion nonconformities
% LocalWords:  macroexpansions
